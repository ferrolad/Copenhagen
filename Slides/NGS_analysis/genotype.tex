
\section{Genotype calling}

\frame{\frametitle{Genotype calling}

	\begin{columns}
	
    		\column{0.6\textwidth}   

		\begin{center}
			\begin{tabular}{c | c}
				Genotype & Likelihood (log10)\\
    				\hline
	    			AA & -2.49\\
    				AC & -3.38\\
    				AG & -1.22\\
    				AT & -3.38\\
            			CC & -9.91\\
            			CG & -7.74\\
            			CT & -9.91\\
            			GG & -7.44\\
            			GT & -7.74\\
            			TT & -9.91\\
        			\hline
			\end{tabular}
       		\end{center}

		\column{0.4\textwidth}

		AAAG \& $\epsilon=0.01$

		\begin{block}{}
			What is the genotype here?
		\end{block}

	\end{columns}


}


\frame{\frametitle{Genotype calling}

	\begin{columns}
	
    	\column{0.6\textwidth}   

		\begin{center}
			\begin{tabular}{c | c}
			Genotype & Likelihood (log10)\\
    		\hline
	    	AA & -2.49\\
    		AC & -3.38\\
    		\textbf{AG} & \textbf{-1.22}\\
    		AT & -3.38\\
            CC & -9.91\\
            CG & -7.74\\
            CT & -9.91\\
            GG & -7.44\\
            GT & -7.74\\
            TT & -9.91\\
        	\hline
			\end{tabular}
       	\end{center}

		\column{0.4\textwidth}

		AAAG \& $\epsilon=0.01$

		What is the genotype? AG.

		\begin{block}{Maximum Likelihood}
			The simplest genotype caller: choose the genotype with the highest likelihood.
		\end{block}

	\end{columns}


}


\frame{\frametitle{Major and minor alleles}

	\begin{block}{Likelihood function}
		\begin{equation*}
			\log P(D|G=A) = \sum_{i=1}^R \log L_{A_j,i}
		\end{equation*}
	\end{block}

	AAAG \& $\epsilon=0.01$

	\begin{center}
		\begin{tabular}{c | c}
		Allele & Likelihood\\
    	\hline
	    \textbf{A} & \textbf{-2.49}\\
    	C & -3.38\\
    	\textbf{G} & \textbf{-1.22}\\
    	T & -3.38\\
        \hline
		\end{tabular}
	\end{center}

	We can reduce the genotype space to 3 entries (from 10, for diploids).
    
}    
    

\frame{\frametitle{Genotype calling}

	AAAG \& $\epsilon=0.01$ \& A,G alleles

	\begin{center}
		\begin{tabular}{c | c}
			Genotype & Likelihood\\
    			\hline
	    		AA & -5.73\\
    			AG & -2.80\\
    			GG & -17.12\\
        		\hline
		\end{tabular}
	\end{center}

       	\vskip 0.5cm
	At what extent is the data affecting the called genotype and its \textbf{confidence}?

	Open jupyter-notebook.
    
}   

\frame{\frametitle{Genotype likelihood ratio}

	\begin{equation*}
		\log_{10} \frac{L_{G(1)}}{L_{G(2)}} > t
	\end{equation*}

	i.e. $t=1$ meaning that the most likely genotype is 10 times more likely than the second most likely one

	Pros and cons?
 	\begin{itemize}
 		\item Yes: \pause genotype are called with higher \textbf{confidence}
    		\item No: \pause more \textbf{missing} data
 	\end{itemize}

	Practical: genotype likelihoods and (basic) genotype calling \\
        \small{\url{https://github.com/mfumagalli/Copenhagen}}

}


\frame{\frametitle{The monster dilemma}

	\begin{figure}[!ht]
		\centering
		\includegraphics[width=4cm]{Pics/LochNessMonster.jpg}
		\caption{Nessie, the Loch Ness Monster. True or fake news?}
	\end{figure}

}


\frame{\frametitle{The monster dilemma - likelihood}

	Let's denote $D$ (data) as the set of observations specifying whether I tell you that I saw
	Nessie ($D=1$) or not ($D=0$).
	
       	$D$ is our sample space, the set of all possible outcomes of the experiment, and $D = \{0,1\}$.

       	We want to make some inferences on the probability that Nessie exists, or that it is true that I saw it (her?).
	Let's denote this probability as $N$.

	\begin{itemize}
		\item $D = \{0,1\}$, whether I tell you I saw Nessie or not.
		\item $N = \{0,1\}$, whether Nessie exists or not.
	\end{itemize}

}

\begin{frame}{The monster dilemma - likelihood}

	\begin{block}{Questions}
		\begin{itemize}
			\item What are $p(D=1|N=1)$ and $p(D=1|N=0)$?
			\item What is a Maximum Likelihood Estimate of $N$?
			\item What is a statistical test for $N=1$?
		\end{itemize}
	\end{block}

\end{frame}

\begin{frame}{The monster dilemma - likelihood}

	Let's assume that $p(D=1|N=0)=0.01$ and $p(D=1|N=1)=0.90$ are valid for each observer $l$, with $l=3$.

	Then the log-likelihood of $N=0$ is given by $\sum_{l=1}^3 log(p(D=1|N=0))=-6.91$ while
	the log-likelihood of $N=1$ is given by $\sum_{l=1}^3 log(p(D=1|N=1))=-0.32$.

	With 3 observations of $D=1$ we obtained a likelihood ratio (LR, of $N=1$ vs $N=0$) of $6.59$.

	\begin{block}{}
		Does the Loch ness monster exist?
	\end{block}

\end{frame}

\frame{\frametitle{"Eyes" thinking}

       	What's "wrong"?\\
	Our inference on $N$, our parameter, is driven solely by our observations, given by our likelihood function.

	\begin{figure}[!ht]
		\centering
		\includegraphics[width=5cm]{Pics/EyeOnly.png}
		\caption{The eye: a "likelihood" organ.}
	\end{figure}

}


\frame{\frametitle{"Blind Brain" thinking}

	In real life we take many decisions based not only on what we observe but also on some believes of ours*.

	\begin{figure}[!ht]
		\centering
		\includegraphics[width=5cm]{Pics/EyeBrain.png}
		\caption{The brain: a "non-likelihood" organ.}
		\label{Fig:EyeBrain}
	\end{figure}

	\small{* unfortunately in many cases}

}


\frame{\frametitle{Eyes+Brain thinking}

	\begin{itemize}
		\item with "eyes only" our intuition is that $p(N|D) \approx p(D|N)$
		\item with "the brain" our intuition is that $p(N|D) \approx p(D|N)p(N)$
	\end{itemize}

	Our "belief" expresses the probability $p(N)$ \textbf{unconditional} of the data.

	\begin{block}{Question}
		How can we define $p(N)$?
	\end{block}

}


\frame{\frametitle{"Eyes + Blind Brain"thinking}

	\begin{block}{}
		The "belief" function $p(N)$ is called \textbf{prior probability} and the joint product of the likelihood $p(D|N)$ and the prior is proportional to the \textbf{posterior probability} $p(N|D)$.
	\end{block}

	\begin{block}{}
		The use of posterior probabilities for inferences is called Bayesian statistics.
	\end{block}

}

\frame{\frametitle{Bayesian \textit{vs.} Likelihoodist}

	\begin{itemize}
		\item we derive "proper" probability distributions of our parameters rather than deriving a point estimate;
		\item a probability is assigned to a hypothesis rather than a hypothesis is tested;
		\item we can "accept" the null hypothesis rather than "fail to reject" it;
		\item parsimony imposed in model choice rather than correcting for multiple tests.
	\end{itemize}

}

\frame{\frametitle{Bayesian inference}

	\begin{figure}[!ht]
		\centering
		\includegraphics[width=8cm]{Pics/bayesian.png}
	\end{figure}

}

\frame{\frametitle{Bayes' Theorem}

	\begin{equation*}
		p(G|D) = \frac{f(D|G)\pi(G)}{\int f(D|G) \pi(G) dG}
	\end{equation*}

	\begin{itemize}
		\item $G$ is not a fixed parameter but a random quantity with prior distribution $\pi(G)$
		\item $p(G|D)$ is the posterior probability distribution of $G$
		\item ${\int p(G|D) dG} = 1$
	\end{itemize}

}


\frame{\frametitle{Genotype posterior probability}

	AAAG \& $\epsilon=0.01$ \& A,G alleles

	\begin{center}
		\begin{tabular}{c | c | c | c}
			Genotype & Likelihood (log) & Prior & Posterior\\
    			\hline
	    		AA & -5.73 & \pause 1/3 & \pause 0.05\\
    			AG & -2.80 & \pause 1/3 & 0.95\\
    			GG & -17.12 & \pause 1/3 & ~0\\
        		\hline
		\end{tabular}
	\end{center}

       	What is the called genotype? What's its confidence?\\
       	\pause
	Only call genotypes if the largest probability is above a certain threshold (e.g. 0.95).

}


\frame{\frametitle{Genotype posterior probability}

	AAAG \& $\epsilon=0.01$ \& A,G alleles \& \textbf{A is the reference allele}

	$P(AA) > P(AG) > P(GG)$

	\begin{center}
		\begin{tabular}{c | c | c | c}
			Genotype & Likelihood (log) & Prior & Posterior\\
    			\hline
	    		AA & -5.73 & \pause 0.80 & 0.22\\
    			AG & -2.80 & \pause 0.15 & 0.78\\
    			GG & -17.12 & \pause 0.05 & ~0\\
        	\hline
		\end{tabular}
	\end{center}
    
	\small{
	Warning:
	the reference allele is just one of the possible alleles, often chosen arbitrarily: why so much weight???
	}

}


\frame{\frametitle{Genotype posterior probability}

	AAAG \& $\epsilon=0.01$ \& A,G alleles \& \textbf{$f(A)=0.7$} from a reference panel

        $P(AA)=?$\\
    	$P(AG)=?$\\
    	$P(GG)=?$

        \begin{center}
                \begin{tabular}{c | c | c | c}
                Genotype & Likelihood (log) & Prior & Posterior\\
        	\hline
            	AA & -5.73 & \pause 0.49 & 0.06\\
        	AG & -2.80 & \pause 0.42 & 0.94\\
        	GG & -17.12 & \pause 0.09 & ~0\\
        	\hline
                \end{tabular}
        \end{center}
	If the assumption of Hardy Weinberg Equilibrium can be reasonably met.
	
	What happens if that's not the case?\\
       	\pause
	Inbreeding can be incorporated: $f_{AA} = (1-f)^2 + (1-f)fF$ ...

}


\frame{\frametitle{"Eyes + non-Blind Brain" inference}

	\begin{figure}[!ht]
		\centering
		\includegraphics[width=8cm]{Pics/statEB.png}
	\end{figure}
       	\centering
       	\small{Empirical Bayesian}

}


\frame{\frametitle{Genotype posterior probability}

	AAAG \& $\epsilon=0.01$ \& A,G alleles \& \textbf{$f(A)=0.7$} from the data itself

	\begin{center}
		\begin{tabular}{c | c | c | c}
			Genotype & Likelihood (log) & Prior & Posterior\\
    			\hline
	    		AA & -5.73 & \pause 0.49 & 0.04\\
    			AG & -2.80 & 0.42 & 0.96\\
    			GG & -17.12 & 0.09 & ~0\\
        		\hline
		\end{tabular}
	\end{center}

	\begin{itemize}
		\item if the assumption of HWE(+-F) can be met (no population structure)
    		\item if enough samples to have a robust estimate of the allele frequencies
	\end{itemize}

	Practical: (advanced) genotype calling\\
	\url{https://github.com/mfumagalli/Copenhagen}
	

}


\frame{\frametitle{Genotype posterior probability}

        AAAG \& $\epsilon=0.01$ \& A,G alleles \& \textbf{$f(A)=0.7$} from the data itself

        \begin{center}
                \begin{tabular}{c | c | c | c}
                Genotype & Likelihood (log) & Prior & Posterior\\
        	\hline
            	AA & -5.73 & 0.49 & 0.04\\
        	AG & -2.80 & 0.42 & 0.96\\
        	GG & -17.12 & 0.09 & ~0\\
        	\hline
                \end{tabular}
        \end{center}

       	\begin{block}{}
        How can we estimate allele frequencies from NGS data?
	\end{block}
}

