
\section{Genotype likelihoods}

\frame{\frametitle{\small{Given a possible genotype, what is the probability of observing this NGS data?}}

	\begin{figure}
		\includegraphics[scale=0.35]{Pics/mappedReads}
	\end{figure}

}


\frame{\frametitle{Genotype likelihoods - equation}

	\begin{block}{Likelihood}
		$P(D|G=\{A_1,A_2,...,A_n\})$
		\newline
		with
        	\newline
		$A_i \in \{A,C,G,T\}$
        	and $n$ being the ploidy level
	\end{block}

	\vskip 1cm
       	\pause
	How many genotypes likelihoods do we need to calculate for each each diploid individual at each site?

}


\frame{\frametitle{Genotype likelihoods - rationale}

	\begin{figure}
		\includegraphics[scale=0.3]{Pics/genoLikes}
	\end{figure}

}


\frame{\frametitle{Genotype likelihoods - calculation}

	\begin{block}{Likelihood function}

		\begin{equation*}
			P(D|G=\{A_1,A_2,...,A_N\}) = \prod_{i=1}^R \sum_{j=1}^N \frac{L_{A_j,i}}{N}
		\end{equation*}
 
 		\begin{itemize}
    			\item $L_{A_j,i} = P(D|A_G=A_j)$
    			\item $A_i \in \{A,C,G,T\}$
			\item $R$ is the depth (nr. of reads)
    			\item $N$ is the ploidy level (nr. of chromosomal copies)
		\end{itemize}
    
	\end{block}

}

\begin{frame}{Genotype likelihoods - example}

	\begin{figure}
                \includegraphics[scale=0.2]{Pics/mappedReads}
        \end{figure}

	\centering
	\small{
        A\\
        A\\
        A\\
        G
        }\\
	with all with quality scores equal to 20 (in phred score)

	\vskip 0.5cm
	\pause
	\centering
	What is $P(D|G=AC) = ?$

\end{frame}


\frame{\frametitle{Genotype likelihoods - example}

	\begin{block}{Likelihood function}

		\begin{equation*}
			P(D|G=\{A_1,A_2,...,A_N\}) = \prod_{i=1}^R \sum_{j=1}^N \frac{L_{A_j,i}}{N}
		\end{equation*}
  
	\end{block}

	\small{
	A\\
	A\\
	A\\
	G\\
	\& Q=20
	}

	\begin{equation*}
		P(D|G=\{A,C\}) = ...
	\end{equation*}

}


\frame{\frametitle{Genotype likelihoods - example}

	\begin{block}{Likelihood function}

		\begin{equation*}
			P(D|G=\{A_1,A_2,...,A_N\}) = \prod_{i=1}^R \sum_{j=1}^N \frac{L_{A_j,i}}{N}
		\end{equation*}
  
	\end{block}

	\small{
	A\\
	A\\
	A\\
	G\\ 
	\& Q=20
	}\\
	$N=2; i=1; A_1=A; A_2=C$

	\begin{equation*}
		P(D|G=\{A,C\}) = (\frac{L_{A,1}}{2} + \frac{L_{C,1}}{2}) \times ...
	\end{equation*}

	What are $L_{A,1}$ and $L_{C,1}$?

}


\frame{\frametitle{Genotype likelihoods - example}

       	\small{
	\begin{block}{Likelihood function}

		\begin{equation*}
			P(D|G=\{A_1,A_2,...,A_N\}) = \prod_{i=1}^R \sum_{j=1}^N \frac{L_{A_j,i}}{N}
		\end{equation*}
  
	\end{block}
	}

        \tiny{
        A\\
        A\\
        A\\
        G
        }

	\begin{equation*}
		L_{C,1} = \pause \frac{\epsilon}{3}	
	\end{equation*}

	\begin{equation*}
    		L_{A,1} = \pause 1-\epsilon
	\end{equation*}
    
	\begin{equation*}
		P(D|G=\{A,C\}) = (\frac{1-\epsilon}{2} + \frac{\epsilon}{6}) \times ...
	\end{equation*}

}


\frame{\frametitle{Genotype likelihoods - example}

       	\small{
	\begin{block}{Likelihood function}

		\begin{equation*}
			P(D|G=\{A_1,A_2,...,A_N\}) = \prod_{i=1}^R \sum_{j=1}^N \frac{L_{A_j,i}}{N}
		\end{equation*}
  
	\end{block}
	}

        \tiny{
        A\\
        A\\
        A\\
        G
        }

	\begin{equation*}
		L_{C,4} = \pause \frac{\epsilon}{3}	
	\end{equation*}

	\begin{equation*}
    		L_{A,4} = \pause \frac{\epsilon}{3}
	\end{equation*}
    
	\begin{equation*}
		P(D|G=\{A,C\}) = (\frac{1-\epsilon}{2} + \frac{\epsilon}{6})^3  \times \frac{\epsilon}{3}
	\end{equation*}

	What is $\epsilon$?

}

\frame{\frametitle{Genotype likelihoods - example}

	\begin{columns}
	
    	\column{0.6\textwidth}   

		\begin{center}
			\begin{tabular}{c | c}
			Genotype & Likelihood (log10)\\
    		\hline
	    	AA & -2.49\\
    		\textbf{AC} & \textbf{-3.38}\\
    		AG & -1.22\\
    		AT & -3.38\\
        	CC & -9.91\\
      		CG & -7.74\\
        	CT & -9.91\\
         	GG & -7.44\\
          	GT & -7.74\\
        	TT & -9.91\\
        	\hline
			\end{tabular}
       	\end{center}

		\column{0.4\textwidth}

		A\\A\\A\\G\\$\epsilon=0.01$

	\end{columns}


}



